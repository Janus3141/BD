
\documentclass[a4paper]{report}
\usepackage{amsmath}
\usepackage{graphicx}
\usepackage{float}
\renewcommand{\baselinestretch}{1}
\usepackage[margin=2cm]{geometry}
\usepackage{proof}
\usepackage{amssymb}
\usepackage{listings}
\usepackage{fancyhdr}
\renewcommand{\familydefault}{\sfdefault}
\lstset{language=Python,
        basicstyle=\small,
        showspaces=false,
        showstringspaces=false}
\pagestyle{fancy}
\lhead{Trabajo pr\'actico 4}
\rhead{Bruno Sotelo, Giuliano Guaragna}

\setlength{\parindent}{0in}

\begin{document}

\begin{titlepage}
\centering
\begin{figure}[H]
    \begin{center}
        \includegraphics[width=2cm,height=2cm]{UNR_logo.jpg}
    \end{center}
\end{figure}
{\scshape\large Facultad de Ciencias Exactas, Ingenier\'ia y Agrimensura\\*
                 Licenciatura en Ciencias de la Computaci\'on\par}
\vspace{5cm}
{\scshape\LARGE Teor\'ia de Bases de Datos \par}
{\huge\bfseries Trabajo Pr\'actico 4 \par}
\vspace{3cm}
{\Large Sotelo, Bruno\par}
{\Large Guaragna, Giuliano\par}
\vfill
{\large 14  / 11 / 2017 \par}
\end{titlepage}

\section*{Gu\'ia de uso del programa}
\subsection*{Lenguaje}
El programa fue implementado utilizando la versi\'on 5.2 de Python 3,
aunque esperamos que funcione para cualquier versi\'on de Python 3.

\subsection*{Formato de entrada}
Preparamos al software para que acepte como entrada un archivo que
contenga varios set de prueba. Cada set estar\'a separado por un
caracter $;$, y cada uno deber\'ia definir un conjunto de atributos ($R$) 
y un conjunto de dependencias ($F$), y opcionalmente un subconjunto de $R$
nombrado $A$. Cada uno de estos conjuntos debe darse como:
$$ A|F|R = \{[[elemento,] elemento]\} $$
Los elementos de $A$ y $R$ deben ser caracteres alfab\'eticos, mientras
un elemento de $F$ ser\'a de la forma $ C1 \to C2 $, donde $C1$ y $C2$
son conjuntos de caracteres alfab\'eticos sin ninguna separaci\'on
particular (aunque pueden separarse por espacios). \\
Entre el c\'odigo entregado incluimos un archivo $test$ donde se
encuentran todos los test propuestos para este software.

\subsection*{Uso de $main.py$}
El programa $main.py$ est\'a preparado para aceptar un archivo con sets de
prueba desde la l\'inea de comandos. Para cada set que especifique s\'olo 
los conjuntos $R$ y $F$ se calcular\'a el conjunto clausura de $F$
($F^{+}$), aunque s\'olo se devuelve la cantidad de elementos de $F^{+}$,
y las claves candidatas sobre $R$. Si el set adem\'as especifica $A$,
se calcula su cierre $A^{+}$ y las claves candidatas de $R$.

\subsection*{Funciones en $dependencies.py$}
\begin{description}
\item [powerset] Toma un objeto iterable y devuelve otro que contiene
                el conjunto de partes del primero, donde cada elemento
                es una tupla.
\item [deps\_closure] Deber\'ia recibir como argumentos un set ($attrs$)
                con los atributos de la relaci\'on y otro set ($deps$)
                conteniendo tuplas que representan las dependencias
                funcionales de la relaci\'on. Si la tupla $(A,B)$ es
                un elemento de $deps$, entonces $B$ depende funcionalmente
                de $A$. Ambos $A$ y $B$ deber\'ian ser de tipo
                $frozenset$. En un principio estos ten\'ian tipo $set$,
                pero lo descartamos por no ser objetos hasheables (por lo
                cual no pueden ser elementos de otro $set$), mientras que
                los $frozenset$ s\'i son hasheables por defecto y
                mantienen casi todas las propiedades de $set$. El
                resultado de la funci\'on ser\'a un $set$ de dependencias
                funcionales, del mismo tipo del $set$ de dependencias de
                la entrada, y es el conjunto clausura de $deps$.
\item [attrs\_closure] Al igual que $deps\_closure$ toma un $set$ de
                atributos de la relaci\'on, y un $set$ con sus
                dependencias funcionales, con el mismo tipo que en
                $deps\_closure$. El resultado es un $set$ de atributos
                (caracteres), que es el conjunto de clausura de $attrs$.
\item [keys] Calcula el conjunto de claves candidatas de la relaci\'on
            dada. El resultado es un $set$ de $frozenset$, donde
            cada $frozenset$ es una clave candidata. Los argumentos
            son dados como en las funciones anteriores.
\end{description}

\subsection*{R\'apida explicaci\'on de $parser.py$}
Para parsear los archivos como se especific\'o anteriormente,
implementamos la clase $Parser$. Para parsear un archivo:
\begin{itemize}
\item Se obtienen los contenidos del archivo (datos en bruto).
\item Se crea un objeto $Parser$ con la funci\'on $Parser$, a la cual
    se le debe pasar los contenidos del archivo como par\'ametro.
\item Se usa el m\'etodo $parse$ de $Parser$ para obtener una lista de
    listas, donde cada sublista es un set de prueba dado en el archivo.
    En cada una de estas se encuentra: en la primera posici\'on el
    conjunto de atributos de la relaci\'on, en la segunda el conjunto de
    dependencias funcionales, y, si se especific\'o el conjunto $A$, en
    la tercera un subconjunto de los atributos de la primera.
\end{itemize}
Si ocurre alg\'un error durante el parseo del archivo, se levantar\'a
una excepci\'on $ParseError$.


\pagebreak


\section*{Ejercicio 3: C\'alculo de claves candidatas}
Observemos que por definici\'on una clave candidata es un subconjunto
de atributos tal que su clausura es igual al conjunto de atributos y es
m\'inima. Con esto en mente ideamos lo siguiente: tomamos un conjunto
que ser\'a devuelto como resultado, inicialmente el conjunto vac\'io.
Se calcula el conjunto de partes del conjunto de atributos, y se lo ordena
de menor a mayor en cantidad de elementos de cada subconjunto. En este
orden se calculan las clausuras de cada subconjunto, as\'i nos aseguramos
de obtener primero los m\'inimos. Si su clausura es igual al conjunto de
atributos, y si en el conjunto resultado no se encuentra uno de sus
subconjuntos, entonces el subconjunto tomado es clave candidata. \\
Inmediatamente se puede definir el siguiente pseudoc\'odigo:
\begin{align*}
    &resultado := \emptyset \\
    &\textbf{for}\; ss \in \mathcal{P}(R) \\
    &\hspace{.5cm} \textbf{if}\; ss^{+} = R \wedge \neg
                 (\exists ss' \in resultado \wedge ss' \subset ss) \\
    &\hspace{.5cm} \textbf{then}\; resultado := resultado \cup \{ss\} \\
    &\textbf{return}\; resultado
\end{align*}
En este c\'odigo falta especificar el hecho de que el conjunto
$\mathcal{P}(R)$ debe iterarse en orden, pero esto se soluciona
f\'acilmente en Python. En el archivo $dependencies.py$ se encuentra
implementado este algoritmo con el nombre de $keys$.

\section*{Salida de $main.py$ para el archivo $test$}
\begin{verbatim}
Set 1:
R: {'B', 'C', 'A', 'D'}
F: {B->DA, BC->A, A->B}
Cardinality(F+): 137
Candidate keys: {B, C} {C, A} 

Set 2:
R: {'E', 'D', 'A', 'B', 'C', 'F'}
F: {BD->FE, BA->C}
Cardinality(F+): 1081
Candidate keys: {B, D, A} 

Set 3:
R: {'I', 'E', 'H', 'G', 'D', 'A', 'B', 'J', 'C', 'F'}
F: {H->J, A->I, DA->HG, BD->FE, BA->C}
A: {'B', 'D'}
A+: {'F', 'B', 'D', 'E'}
Candidate keys: {B, D, A} 

Set 4:
R: {'E', 'H', 'G', 'D', 'A', 'B', 'C', 'F'}
F: {E->H, D->G, H->E, A->BC, E->A, C->D}
A: {'C', 'A'}
A+: {'B', 'C', 'G', 'D', 'A'}
Candidate keys: {F, E} {H, F} 

Set 5:
R: {'E', 'G', 'D', 'A', 'B', 'C', 'F'}
F: {FG->C, A->G, A->F, E->A, D->B, B->E, C->D}
A: {'F', 'G'}
A+: {'B', 'C', 'E', 'G', 'F', 'D', 'A'}
Candidate keys: {C} {E} {B} {F, G} {A} {D} 
\end{verbatim}


\end{document}

